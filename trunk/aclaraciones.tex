\section*{Aclaraciones}

\subsection*{Extension del Secu($\alpha$)}
\paragraph{}
El Tad SecuOrdSinRep($\alpha$) es una extension del Tad Secu($\alpha$) en la cual se agregan algunas operaciones y se reemplaza el generador $\puntito$ por agregarOdenado. En consecuencia, las operaciones hechas sobre el operador puntito pierden validez.

\subsection*{Ordenes de los elementos $\alpha$}
\paragraph{}
Durante este trabajo pr\'actico se di\'o por sentado que se ten\'ian los siguientes \'ordenes totales:

\begin{itemize}
\item En la estructura del enrutador, el campo \textit{reglas} es de tipo secuOrdSinRep. Las tuplas que esta contiene, se ordenan por versi\'on, es decir, se ordenan seg\'un el orden dado por el primer campo de cada tupla.
\item En las secuencias de eventos, se da por sentado que el orden de los mismos est\'a dado por el timestamp, orden\'andose de forma creciente.
\end{itemize}

\subsection*{Conjunto de interfaces}
\paragraph{}
Como se indica en el enunciado, decidimos reemplazar al conjunto de interfaces por una cantidad m\'axima de interfaces, las cuales van de 0 a cantInterfaces-1.

\subsection*{Aliasing Router}

\paragraph{}
En la interfaz del router, los siguientes par\'ametros sean de tipo in, inout o devueltos como resultado, son pasados por referencia:
\begin{itemize}
 \item Router
 \item ReglaDir
 \item Evento
 \item DirIp 
 \item Conjunto
 \item SecuOrdSinRep
\end{itemize}

\paragraph{}
El resto de los par\'ametros, son tipos b\'asicos, por lo cual son pasados por copia.

\paragraph{}
En el caso de los elementos devueltos como resultado, si no son de alguno de los tipos compuestos aclarados anteriormente, siempre se devuelven por copia.

\subsection*{Aliasing SecuOrdSinRep}

\paragraph{}
En la interfaz de la secuOrdSinRep, cuando se recibe un par\'ametro de tipo $\alpha$, si es un tipo b\'asico, se pasa por copia, mientras que si es un tipo complejo, se pasa por referencia.

\paragraph{}
El resto de los par\'ametros, se pueden considerar par\'ametros b\'asicos, por lo cual se pasan por copia.

\paragraph{}
En el caso de los elementos devueltos como resultado, siempre se devuelven por copia.
\subsection*{Aliasing Conjunto}

\paragraph{}
En la interfaz del conjunto, los par\'ametros recibidos como in o inout de tipo Conjunto, son pasados por referencia.

\paragraph{}
En el caso de los elementos param\'etricos, si el par\'ametro es un tipo b\'asico, se pasa por copia, mientras que si es complejo, se pasa por referencia.

\paragraph{}
En el caso de los elementos devueltos como resultado, siempre se devuelven por copia si son de tipos b\'asicos, en caso de devolver un Conjunto, este se devuelve por referencia.

\subsection*{Aliasing Arbol}

\paragraph{}
En la interfaz del arbolDeReglas, los siguientes par\'ametros sean de tipo in o inout, son pasados por referencia:
\begin{itemize}
 \item arbolDeReglas
 \item ReglaDir
 \item DirIp 
\end{itemize}

\paragraph{}
En el caso de los elementos devueltos como resultado, siempre se devuelven por copia.

\subsection*{Descripcion de funciones utilizadas}
\paragraph{}
Las siguientes funciones fueron utilizadas para describir Ordenes de complejidad o funciones de representaci\'on o abstracci\'on. Las mismas est\'an incluidas dentro de los TAD correspondientes, pero no tienen ninguna funci\'on an\'aloga dentro de los m\'odulos presentados para este trab\'ajo pr\'actico. Adem\'as, la funci\'on altura solo se utiliza para la funci\'on de representaci\'on del Enrutador y romper\'ia la igualdad observacional del TAD arbolDeReglas, por lo cual la misma se encuentra axiomatizada en este apartado.


\alinearfuncs{posicionRelativa}{\hspace{5cm}}
	\func{posicionRelativa}{secuOrdSinRep($\alpha$),$\alpha$}{nat}{}
	\func{posicionRelativa}{conj($\alpha$),$\alpha$}{nat}{}
	\func{altura}{ArbolDeReglas}{nat}{}

\axioma{altura(nuevo())}{0}
\axioma{altura(agRegla(a,r))}{max(cantBits(r),altura(r))}
