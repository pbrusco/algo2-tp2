\section*{Aclaraciones}

\subsection*{Extension del Secu($\alpha$)}
\paragraph{}
El Tad SecuOrdSinRep($\alpha$) es una extension del Tad Secu($\alpha$) en la cual se agregan algunas operaciones y se reemplaza el generador $\puntito$ por agregarOdenado. En consecuencia, las operaciones hechas sobre el operador puntito pierden validez.

\subsection*{Ordenes de los elementos $\alpha$}
\paragraph{}
Durante este trabajo pr\'actico se di\'o por sentado que se ten\'ian los siguientes \'ordenes totales:

\begin{itemize}
 \item Un evento es una tupla, representada de la siguiente forma: \\
tupla $\langle$ caida?: bool  $\times$ timestamp: nat $\times$ inter: interfaz  $\rangle$.\\
Un evento EV1 se considera mayor que otro evento EV2 si EV1.timestamp $<$ EV2.timestamp.
\item En la estructura del enrutador, el campo \textit{reglas} es de tipo secuOrdSinRep. Las tuplas que esta contiene, se ordenan por versi\'on, es decir, se ordenan seg\'un el orden dado por el primer campo de cada tupla.
\end{itemize}

\subsection*{Aliasing Router}

\paragraph{}
En la interfaz del router, los siguientes par\'ametros sean de tipo in, inout o devueltos como resultado, son pasados por referencia:
\begin{itemize}
 \item Router
 \item ReglaDir
 \item Evento
 \item DirIp 
 \item Conjunto($\alpha$)
 \item SecuOrdSinRep($\alpha$)
\end{itemize}

\paragraph{}
El resto de los par\'ametros, son tipos b\'asicos, por lo cual son pasados por copia.

\paragraph{}
En el caso de los elementos devueltos como resultado, si no son de alguno de los tipos compuestos aclarados anteriormente, siempre se devuelven por copia.

\subsection*{Aliasing SecuOrdSinRep}

\paragraph{}
En la interfaz de la secuOrdSinRep, cuando se recibe un par\'ametro de tipo $\alpha$, si es un tipo b\'asico, se pasa por copia, mientras que si es un tipo complejo, se pasa por referencia.

\paragraph{}
El resto de los par\'ametros, se pueden considerar par\'ametros b\'asicos, por lo cual se pasan por copia.

\paragraph{}
En el caso de los elementos devueltos como resultado, siempre se devuelven por copia si son de tipos b\'asicos, en caso de devolver una Secuencia, esta se devuelve por referencia.
\subsection*{Aliasing Conjunto}

\paragraph{}
En la interfaz del conjunto, los par\'ametros recibidos como in o inout de tipo Conjunto, son pasados por referencia.

\paragraph{}
En el caso de los elementos param\'etricos, si el par\'ametro es un tipo b\'asico, se pasa por copia, mientras que si es complejo, se pasa por referencia.

\paragraph{}
En el caso de los elementos devueltos como resultado, siempre se devuelven por copia.

\subsection*{Aliasing Arbol}

\paragraph{}
En la interfaz del arbolDeReglas, los siguientes par\'ametros sean de tipo in o inout, son pasados por referencia:
\begin{itemize}
 \item arbolDeReglas
 \item ReglaDir
 \item DirIp 
\end{itemize}

\paragraph{}
En el caso de los elementos devueltos como resultado, siempre se devuelven por copia.

\subsection*{Descripcion de funciones utilizadas}
\paragraph{}
Las siguientes funciones fueron utilizadas para describir Ordenes de complejidad o funciones de representaci\'on o abstracci\'on:


\alinearfuncs{posicionRelativa}{\hspace{5cm}}	
\alinearaxiomas{ordenada(agregarOrdenado(a,s)}{\hspace{3cm}}


		\func{ordenada}{secuOrdSinRep($\alpha$)}{bool}{}

		\axioma{ordenada($<>$)}{true}
		\axioma{ordenada(agregarOrdenado(a,s))}{(a $<_\alpha$ prim(s))) $\wedge$ ordenada(s))}
		
		\vspace{11pt}		

		\func{posicionRelativa}{secuOrdSinRep($\alpha$),$\alpha$}{nat}{}

		\axioma{posicionRelativa(s,a)}{\lif (vacia?(s) \oluego  a $\leq_\alpha$ prim(s)) \lthen 0 \\ \hspace*{-0.87cm} \lelse 1 + posicionRelativa(fin(s),a)}

		\vspace{11pt}		

		\func{posicionRelativa}{conj($\alpha$),$\alpha$}{nat}{}

		\axioma{posicionRelativa(c,a)}{\lif ($\emptyset$?(c))  \lthen 0 \\ \hspace*{-0.87cm} \lelse \lif (a $\leq_\alpha$ dameUno(c)) \lthen posicionRelativa(sinUno(c),a) \\ \hspace*{-0.87cm} \lelse 1 + posicionRelativa(sinUno(c),a)}


		\vspace{11pt}		

		\func{altura}{ArbolDeReglas}{nat}{}
		\axioma{altura(nuevo())}{0}
		\axioma{altura(agRegla(a,r),d)}{max(cantBits(r),altura(r))}