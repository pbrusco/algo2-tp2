\section*{Modulo \nombretad{ArbolDeReglas}}

%INTERFAZ
\subsection*{Interfaz}
\textbf{interfaz} \textit{arbolDeReglas}\\
\textbf{se explica con la especificaci\'on de} \textit{arbolDeReglas}\\
\textbf{usa interfaces} \nombretad{Bool}, \nombretad{Nat}, \nombretad{dirIp},\nombretad{ReglaDir}\\
\textbf{genero} \textit{abr}\\

\textbf{operaciones}\\

$Crear$() $\longrightarrow res$: abr \hfill (O(1)) \\
$\{$true$\}$\\
$\{\sombreritoWide{res} \igobs nuevo()\}$\\

$agRegla$(inout $a$: abr, in $r$: regla) \hfill (O(version(r))) \\
$\{\sombreritoWide{a} \igobs\ a_0\}$\\
$\{\sombreritoWide{a} \igobs\ agRegla(a_0,\sombreritoWide{r})\}$\\

$interfazDeSalida$(in $a$: abr, in $d$: dirIp) $\longrightarrow res$: nat \hfill (O(version(d)))) \\
$\{tieneRegla?(\sombreritoWide{a},\sombreritoWide{d})\}$\\
$\{\sombreritoWide{res}\igobs$ $interfazDeSalida(\sombreritoWide{a},\sombreritoWide{d})$\}\\


%REPRESENTACION
\subsection*{Representaci\'on}
\textit{arbolDeReglas} \textbf{se representa con} \textit{puntero(estr)}\\
\textbf{donde} \textit{estr} \textbf{es}\\
\textit{tupla$\langle$ inter: puntero(interfaz)\\
\hspace*{1.1cm}$\times$ izq: puntero(estr)\\
\hspace*{1.1cm}$\times$ der: puntero(estr) $\rangle$}
\vspace{33pt}


%INVARIANTE DE REPRESENTACION
\subsection*{Invariante\ de \ Representaci\'on}
\alinearfuncs{Rep}{\hspace{3cm}}
\func{Rep}{puntero(estr)}{boolean}{}
\vspace{11pt}
($\forall$ p: puntero(estr)) Rep(p) = true $\Leftrightarrow$ no existen ciclos
\vspace{33pt}


%FUNCION DE ABSTRACCION
\subsection*{Funci\'on de Abstracci\'on}

\func{Abs}{puntero(estr)/p}{abr}{Rep(p)}
\vspace{11pt}
($\forall$ p: puntero(estr)) Abs(p) = a: abr/ \\
\\
($\forall$ d: dirIp) ((tieneRegla(a,d) $=$ existeRegla(p,pasarABits(d),0) $\yluego$\\
\\
(tieneRegla(a,d) $\impluego$ (interfazDeSalida(a,d) = dameSalida(p,pasarABits(d),0,null)))))
\vspace{33pt}
\alinearfuncs{existeReglala}{\hspace{11.2cm}}
\alinearaxiomas{dameSalida(p,a,n,pi)}
\func{existeRegla}{puntero(estr)/p,arregloDeBool, nat}{bool}{Rep(p)}
\axioma{existeRegla(p,a,n)}{\lif n $=$ tam(a) $\vee$ p $=$ null \lthen \false\\
			    \hspace*{-0.85cm}\lelse \lif (*p).inter $=$ null \lthen\\
			    \hspace*{1cm}\lif a[n] $=$ \true \lthen existeRegla((*e).der,a,n+1)\\
			    \hspace*{1cm}\lelse existeRegla((*e).izq,a,n+1)\\
			    \hspace*{1cm}\lfi\\
			    \hspace*{-0.85cm}\lelse \true\\
			    \hspace*{-0.85cm}\lfi}

\vspace{15pt}

\func{dameSalida}{puntero(estr)/p, arregloDeBool/a, nat, puntero(estr)}{interfaz}{Rep(p) $\yluego$ existeRegla(p,a,0)}
\axioma{dameSalida(p,a,n,pi)}{\lif n $=$ tam(a) $\vee$ p $=$ null \lthen *pi\\
			      \hspace*{-0.85cm}\lelse \lif a[n] $=$ true \lthen\\
			      \hspace*{1cm}\lif pi $\neq$ null \lthen dameSalida((*p).der,a,n+1,pi)\\
			      \hspace*{1cm}\lelse dameSalida((*p).der,a,n+1,(*p).int)\\
			      \hspace*{1cm}\lfi\\
			      \hspace*{-0.85cm}\lelse\\
			      \hspace*{1cm}\lif pi $\neq$ null \lthen dameSalida((*p).izq,a,n+1,pi)\\
			      \hspace*{1cm}\lelse dameSalida((*p).izq,a,n+1,(*p).int)\\
			      \hspace*{1cm}\lfi\\
			      \hspace*{-0.85cm}\lfi}
			      
\textit{Aclaraci\'on: dameSalida funciona dado que asumimos que el \'arbol se poda cada vez que agregamos una regla como lo indica el algor\'itmo}

\newpage
%ALGORITMOS
\subsection*{Algoritmos}
 \incmargin{1em}
    \linesnumbered
    \restylealgo{boxed}
    \dontprintsemicolon

\textit{i}crear()   $\longrightarrow$	$res$ : puntero(estr)\\
\begin{algorithm}[H]
\Orden{O(1)}
\BlankLine
res $\leftarrow$ NULL \\
\caption{crear}
\end{algorithm}

\textit{i}agRegla(inout $p$ : puntero(estr),in $r$ : regla)\\
\begin{algorithm}[H]
\Orden{O(version(dirIp(r)))}
\BlankLine
\textbf{var} i : int\\
\textbf{var} pAux : puntero(estr) 	\\
\textbf{var} a : arreglo\_dimensionable de bool\\
\textbf{var} x : Interfaz\\
\BlankLine 
a $\leftarrow$ CrearArreglo(version(dirIp(r))*8) \tcp*{O(version(dirIp(r)))}
\BlankLine
\If {p = NULL \tcp*{O(1)}}{p $\leftarrow$ $\&$ $\langle$NULL,NULL,NULL$\rangle$ \tcp*{O(1)}}
\BlankLine 
a $\leftarrow$ pasarABits(dirIP(r)) \tcp*{O(version(dirIp(r)))}
pAux $\leftarrow$ p \tcp*{O(1)}
i $\leftarrow$ 0 \tcp*{O(1)}
\BlankLine 
\While{i $\neq$ cantBits(r) \tcp*{O(cantBits(r))}}{
	\eIf {a[i] = true \tcp*{O(1)}}{
		\If{(*pAux).der = NULL \tcp*{O(1)}}{(*pAux).der $\leftarrow$ $\&$ $\langle$NULL,NULL,NULL$\rangle$ \tcp*{O(1)}}
			 pAux $\leftarrow$ (*pAux).der \tcp*{O(1)}}{
				 \If{(*pAux).izq = NULL \tcp*{O(1)}}{(*pAux).izq $\leftarrow$ $\&$ $\langle$NULL,NULL,NULL$\rangle$ \tcp*{O(1)}}				 
				 pAux $\leftarrow$ (*pAux).izq \tcp*{O(1)}}

		i $\leftarrow$ i + 1 \tcp*{O(1)}
		}
\BlankLine 
\eIf {(*pAux).inter = NULL \tcp*{O(1)}}{
x $\leftarrow$ inter(r) \tcp*{O(1)}
(*pAux).inter $\leftarrow$ $\&$x \tcp*{O(1)}}{*((*pAux).inter) $\leftarrow$ inter(r)\tcp*{O(1)}}
\BlankLine 
(*pAux).der $\leftarrow$ NULL \tcp*{O(1)}
(*pAux).izq $\leftarrow$ NULL \tcp*{O(1)\footnote{En estas lineas podamos el arbol sin pensar el la etapa implementativa ya que ser\'ia un desperdicio de memoria. En ese momento se resolvera de alguna manera que emule la poda.}}

\end{algorithm}


\textit{i}InterfazDeSalida(in $p$ : puntero(estr),in $d$ : dirIp)   $\longrightarrow$	$res$ : interfaz\\
\begin{algorithm}[H]
\Orden{O(version(d))}
\BlankLine
\textbf{var} i : int\\
\textbf{var} pRes: puntero(interfaz)\\
\textbf{var} pResProvisorio: puntero(interfaz)\\
\textbf{var} pAux puntero(estr) 	\\
\textbf{var} dBits: arreglo dinamico de bool\\
pRes $\leftarrow$ NULL \tcp*{O(1)}
dBits $\leftarrow$ pasarABits(d) \tcp*{O(version(d))}
pResProvisorio $\leftarrow$ NULL \tcp*{O(1)}
pAux $\leftarrow$ p \tcp*{O(1)}
i $\leftarrow$ 0 \tcp*{O(1)}
\BlankLine 
\While{pRes = NULL \tcp*{O(version(d))\footnote{Esta complejidad se obtiene ya que se recorre el \'arbol hasta como mucho 8*version(d) y esto es O(version(d))}}}{
	\If{(pAux$\rightarrow$inter) $\neq$ NULL \tcp*{O(1)}}{pResProvisorio $\leftarrow$ (pAux $\rightarrow$ inter) \tcp*{O(1)}}
	
	\eIf{dBits[i] = true \tcp*{O(1)}}{
		\eIf{(pAux$\rightarrow$der) = NULL \tcp*{O(1)}}{pRes $\leftarrow$ pResProvisorio \tcp*{O(1)}}{pAux $\leftarrow$ (pAux$\rightarrow$der) \tcp*{O(1)}}
	}{
		\eIf{(pAux$\rightarrow$izq) = NULL \tcp*{O(1)}}{pRes $\leftarrow$ pResProvisorio \tcp*{O(1)}}{pAux $\leftarrow$ (pAux$\rightarrow$der) \tcp*{O(1)}}
	}
	i $\leftarrow$  i + 1 \tcp*{O(1)}
}
res $\leftarrow$ *(pRes) \tcp*{O(1)}
\caption{interfazDeSalida}
\end{algorithm}
\newpage

\textit{i}tieneRegla(in $p$ : puntero(estr),in $d$ : dirIp)   $\longrightarrow$	$res$ : bool\\
\begin{algorithm}[H]
\Orden{O(version(d))}
\BlankLine	
\textbf{var} i : int\\
\textbf{var} pAux : puntero(estr) 	\\
\textbf{var} sePuedeSeguir : bool	\\
\textbf{var} dBits: arreglo dinamico de bool\\
\BlankLine
pAux $\leftarrow$ p \tcp*{O(1)}
dBits $\leftarrow$ pasarABits(d) \tcp*{O(version(d))}
i $\leftarrow$ 0 \tcp*{O(1)}
sePuedeSeguir $\leftarrow$ true \tcp*{O(1)}
res $\leftarrow$ false \tcp*{O(1)}
\BlankLine
\If {pAux = NULL \tcp*{O(1)}}{
res $\leftarrow$ false \tcp*{O(1)}
 sePuedeSeguir $\leftarrow$ false \tcp*{O(1)}}
\BlankLine
\BlankLine
\While{res = false $\wedge$ sePuedeSeguir = true \tcp*{O(version(d))\footnote{Esta complejidad se obtiene ya que se recorre el arbol hasta como mucho 8*version(d) y esto es O(version(d))}}}{
	\eIf{dBits[i] = true \tcp*{O(1)}}{
		\eIf{(pAux$\rightarrow$der) = NULL \tcp*{O(1)}}{sePuedeSeguir $\leftarrow$ false \tcp*{O(1)}}{pAux $\leftarrow$ (pAux$\rightarrow$der) \tcp*{O(1)}}
	}{
		\eIf{(pAux$\rightarrow$izq) = NULL \tcp*{O(1)}}{sePuedeSeguir $\leftarrow$ false \tcp*{O(1)}}{pAux $\leftarrow$ (pAux$\rightarrow$izq) \tcp*{O(1)}}
	}
	
	i $\leftarrow$  i + 1 \tcp*{O(1)}
\BlankLine
\If{(pAux$\rightarrow$inter) $\neq$ NULL \tcp*{O(1)}}{res $\leftarrow$ true \tcp*{O(1)}}
}

\end{algorithm}

\newpage
      
\textit{i}pasarABits(in $d$ : dirIp) $\longrightarrow$ $res$ : arreglo\_dinamico de bool\\
\begin{algorithm}[H]
\Orden{O(version(d))}
\BlankLine	
\textbf{var} a : arreglo\_dinamico de bool\\
\textbf{var} i, j, aux : nat	\\
\BlankLine
a $\leftarrow$ crearArrego(8*version(d)) \tcp*{O(version(d))}
i $\leftarrow$ 7 \tcp*{O(1)}
j $\leftarrow$ 0 \tcp*{O(1)}
\BlankLine
\While{(j $<$ version(d)) \tcp*{O(version(d))\footnote{Esto se ejecuta version(d) veces. Esto tambi\'en podr\'ia haberse implementado con un ``for'' en vez de con un ``while''. Por cada iteraci\'on de ``j'' se itera 8 veces con ``i'', por lo que la complejidad total es O(8*version(d)), pero asint\'oticamente es O(version(d))}}}{
	aux $\leftarrow$ d[j] \tcp*{O(1)}
	\BlankLine
	\While{(i $\geq$ 8*j) \tcp*{O(1)\footnote{Esto se ejecuta 8 veces, porque cuando se entra aqu\'i por primera vez vale que i = j+7. Esto tambi\'en podr\'ia haberse implementado con un ``for'' en vez de con un ``while''}}}{
		\eIf{aux mod 2 = 1 \tcp*{O(1)}}{a[i] $\leftarrow$ true \tcp*{O(1)}}{a[i] $\leftarrow$ false \tcp*{O(1)}}
		\BlankLine
		aux $\leftarrow$ aux / 2 \tcp*{O(1)}
		i $\leftarrow$ i $-$ 1 \tcp*{O(1)}
	}	
	\BlankLine
	i $\leftarrow$ i $+$ 15 \tcp*{O(1)}
	j $\leftarrow$ j $+$ 1	 \tcp*{O(1)}
}
res	$\leftarrow$ a \tcp*{O(1)}
\caption{pasarABits}
\end{algorithm}
